\documentclass[a4paper,11pt,headsepline]{scrartcl}
\include{packages}
\include{styles}

\usepackage[ngerman]{babel}
\selectlanguage{ngerman}

\bibliographystyle{alpha}

\hypersetup{
 pdfauthor={Oliver Klee, typo3-coding@oliverklee.de, @oliklee},
 pdftitle={Testgetriebene Entwicklung mit PHPUnit (mit oder ohne TYPO3)},
 pdfsubject={Handout zu Oliver Klees Workshops zu testgetriebener Entwicklung mit PHPUnit (mit oder ohne TYPO3).},
 pdfkeywords={TDD, unit testing, PHPUnit, workshop, seminar, PHP}
}

\author{
  Oliver Klee | \texttt{typo3-coding@oliverklee.de} | \texttt{@oliklee} \\
  \url{https://github.com/oliverklee/tdd-reader}
}

\date{Version 3.1.0, \today, für PHP >= 7.0 und TYPO3 CMS 7.6-8.7}

\title{
  Spickzettel zu testgetriebener Entwicklung mit PHPUnit
}

\begin{document}
\nocite*{}

\maketitle

\section*{Lizenz}

Dieser Reader ist unter einer \emph{Creative-Commons}-Lizenz lizensiert, und zwar konkret unter der \emph{Namensnennung-Weitergabe unter gleichen Bedingungen~4.0 (CC BY-SA 4.0)}. Das bedeutet, dass ihr den Reader unter diesen Bedingungen für euch kostenlos verbreiten, bearbeiten und nutzen könnt (auch kommerziell):

\begin{description}
  \item[Namensnennung.] Ihr müsst den Namen des Autors (Oliver Klee) nennen. Wenn ihr dabei zusätzlich auch noch die Quelle\footnote{\url{https://github.com/oliverklee/tdd-reader}} nennt, wäre das nett. Und wenn ihr mir zusätzlich eine Freude machen möchtet, sagt mir per E-Mail Bescheid.
  \item[Weitergabe unter gleichen Bedingungen.] Wenn ihr diesen Inhalt bearbeitet oder in anderer Weise umgestaltet, verändert oder als Grundlage für einen anderen Inhalt verwendet, dann dürft ihr den neu entstandenen Inhalt nur unter Verwendung identischer Lizenzbedingungen weitergeben.
  \item[Lizenz nennen.] Wenn ihr den Reader weiter verbreitet, müsst ihr dabei auch die Lizenzbedingungen nennen oder beifügen.
\end{description}

Die ausführliche Version dieser Lizenz findet ihr online.\footnote{\url{http://creativecommons.org/licenses/by-sa/4.0/}}


\pagebreak
\tableofcontents

\pagebreak
\section{Test-Level und -Typen}

Dank für viele dieser Konzepte geht an Filip Defar, der in seinem Blog\footnote{\url{http://filipdefar.com/2015/06/tested-be-thy-name.html}} darüber geschrieben hat.


\subsection{Test-Level}

\paragraph{Unit-Tests} sind klein und laufen schnell. Sie sind außerdem normalerweise am schnellsten zu schreiben. Ihr benutzt sie, um alle Details eurer Anwendung zu testen~-- einschließlich Fälle, die auf GUI-Ebene nicht sinnvoll testbar sind. Es gibt einige Ausnahmen, bei denen Unit-Tests nicht sinnvoll sind, z.,B.~Tests für Queries in Extbase-Repositories. Manchmal wird der Begriff \emph{Unit-Tests} auch für alle Tests benutzt, die mit einem Unit-Testing-Framework ausgeführt werden (also auch etwa Systemtests).

\paragraph{Integrationstests} laufen langsamer. Ihr benutzt sie, um zu testen, wie Komponenten zusammenarbeiten, und testet normalerweise nicht jedes Detail damit. In der TYPO3-Welt werden die Integrationstests \fett{funktionale Tests} genannt.

\paragraph{Systemtests} sind die größten und langsamsten Tests. Ihr benutzt sie, um eure Anwendung als Ganzes zu testen. Systemtests werden normalerweise für relativ wenige, große Usecases benutzt.


\subsection{Test-Typen}

\paragraph{Blackbox-Tests} testen, wie sich eine Einheit oder API nach außen verhält. Sie gehen davon aus, dass ihr kein Wissen darüber habt, wie die Einheit innen arbeitet. Blackbox-Tests erleichtern Refactoring. Sie sind das eine Ende eines Kontinuums, bei dem Whitebox-Tests das andere Extrem bilden.

\paragraph{Whitebox-Tests} testen, wie einen Einheit innen arbeitet. Dies erschwert Refactoring und wird nicht empfohlen. Blackbox-Tests erleichtern Refactoring. Sie sind das eine Ende eines Kontinuums, bei dem Blackbox-Tests das andere Extrem bilden. Tests sollten normalerweise Blackbox-Tests sein (oder sehr dunkelgrau).

\paragraph{Funktionale Tests} testen das Verhalten einer Einheit. Die allermeisten Test sind funktionale Tests. Hinweis: In der TYPO3-Welt versteht man unter funktionalen Tests eigentlich \emph{Integrationstests} (und manchmal \emph{Systemtests}).

\paragraph{Akzeptanztests} testen, wie sich eine Anwendung auf GUI-Ebene (oder im Web-Interface) verhält. Manchmal wird dieser Begriff als Synonym für \emph{Story-Tests} benutzt.

\paragraph{Regressionstests} testen, dass Bugs nicht erneut auftreten. Man schreib sie optimalerweise direkt zusammen mit dem Bugfix.

\paragraph{Smoketests} testen, dass die Anwendung überhaupt irgendwie läuft.

\paragraph{Storytests} testen Usecases für User-Stories. Sie werden normalerweise in der Sprache \emph{Gherkin} für Behavior-driven-Design (BDD) geschrieben. Dieser Begriff wird manchmal als synonym für \emph{Akzeptanztests} benutzt.



\section{Testphasen}

Innerhalb der einzelnen Testmethoden sollte Code für die Testphasen durch Leerzeilen getrennt sein, damit die Struktur direkt sichtbar wird.

\hspace{1em}

\begin{tabular}{l l l}
  \emph{Setup}    & (aufbauen)  & \texttt{setUp()} und Code am Anfang der Testmethode (falls nötig)\\
  \emph{Exercise} & (ausführen) & Methodenaufruf \\
  \emph{Verify}   & (prüfen)    & \texttt{self::assert…()} \\
  \emph{Teardown} & (abbauen)   & \texttt{tearDown()} \\
\end{tabular}


\section{PhpStorm-Plugins}

Diese beiden PhpStorm-Plugins sind bei Unit-Tests besonders hilfreich:

\begin{itemize}
  \item PHPUnit Autocomplete Assistant\footnote{\url{https://plugins.jetbrains.com/plugin/7722-phpunit-autocomplete-assistant}}
  \item DynamicReturnTypePlugin\footnote{\url{https://plugins.jetbrains.com/plugin/7251-dynamicreturntypeplugin}}
\end{itemize}





\section{Benennung von Dateien und Klassen}

\subsection{Dateinamen}

\begin{tabular}{|l|l|}
  \hline
  \fett{Dateiname des Produktionscodes} & \fett{Name der Testdatei} \\
  \hline
  \texttt{Classes/Domain/Model/Shoe.php} & \texttt{Tests/Unit/Domain/Model/ShoeTest.php} \\
  \hline
  \texttt{Classes/Service/BaristaService.php} & \texttt{Tests/Unit/Service/BaristaServiceTest.php} \\
  \hline
\end{tabular}


\subsection{Klassennamen}

\small
\begin{tabular}{|l|l|}
  \hline
  \fett{Name der Klasse im Produktionscode} & \fett{Name der Testklasse} \\
  \hline
  \texttt{Shoes\textbackslash Shop\textbackslash Domain\textbackslash Model\textbackslash Shoe} & \texttt{Shoes\textbackslash Shop\textbackslash Tests\textbackslash Unit\textbackslash Domain\textbackslash Model\textbackslash ShoeTest} \\
  \hline
  \texttt{Shoes\textbackslash Shop\textbackslash Service\textbackslash BaristaService} & \texttt{Shoes\textbackslash Shop\textbackslash Tests\textbackslash Unit\textbackslash Service\textbackslash BaristaServiceTest} \\
  \hline
\end{tabular}
\normalsize

\pagebreak
\section{Struktur von Testklassen}

\subsection{Test-Basisklasse}

\subsubsection{Nicht-TYPO3-Projekt}
\paragraph{PHPUnit <= 5.7} \texttt{\textbackslash PHPUnit\_Framework\_TestCase}
\paragraph{PHPUnit >= 6.0} \texttt{\textbackslash PHPUnit\textbackslash Framework\textbackslash TestCase}

\subsubsection{TYPO3-Projekt}
\subsubsection{Mit \texttt{nimut/testing-framework}}

\paragraph{Unit-Tests}
\texttt{\textbackslash Nimut\textbackslash TestingFramework\textbackslash TestCase\textbackslash UnitTestCase}

\paragraph{ViewHelper-Tests}
\texttt{\textbackslash Nimut\textbackslash TestingFramework\textbackslash TestCase\textbackslash ViewHelperBaseTestcase}

\paragraph{Functional-Tests}
\texttt{\textbackslash Nimut\textbackslash TestingFramework\textbackslash TestCase\textbackslash FunctionalTestCase}


\subsubsection{Mit \texttt{typo3/testing-framework}}

\paragraph{Unit-Tests}
\texttt{\textbackslash TYPO3\textbackslash TestingFramework\textbackslash Core\textbackslash Unit\textbackslash UnitTestCase}

\paragraph{Functional-Tests}
\texttt{\textbackslash TYPO3\textbackslash TestingFramework\textbackslash Core\textbackslash Unit\textbackslash FunctionalTestCase}


\subsection{Nicht-TYPO3-PHP-Projekt mit Composer}

Es gibt auf GitHub dazu auch ein leeres Startprojekt:\\
\url{https://github.com/oliverklee/tdd-seed}

\subsubsection{composer.json}

Diese composer.json installiert PHPUnit und vfsStream:

\begin{jsoncode}
{
    "require-dev": {
        "phpunit/phpunit": "^6.3",
        "mikey179/vfsstream": "^1.6"
    },
    "autoload": {
        "psr-4": {
            "..."
        }
    },
    "autoload-dev": {
        "psr-4": {
            "..."
        }
    }
}
\end{jsoncode}

\subsubsection{Testcase}
\begin{phpcode}
namespace OliverKlee\Books\Tests\Unit\Domain\Model;

use OliverKlee\Books\Domain\Model;

class BookTest extends \PHPUnit\Framework\TestCase
{
    /**
     * @var Book
     */
    protected $subject = null;

    protected function setUp()
    {
        $this->subject = new Book();
    }

    /**
     * @test
     */
    public function getTitleInitiallyReturnsEmptyString()
    {
        self::assertSame('', $this->subject->getTitle());
    }

    /**
     * @test
     */
    public function setTitleSetsTitle()
    {
        $title = 'foo bar';

        $this->subject->setTitle($title);

        self::assertSame($title, $this->subject->getTitle());
    }
}
\end{phpcode}


\subsection{Extbase-Extensions}

Es gibt auf GitHub dazu auch ein Beispielprojekt (das \emph{Tea-Example}):\\
\url{https://github.com/oliverklee/ext_tea}

\begin{phpcode}
namespace \OliverKlee\Shop\Tests\Unit\Domain\Model;

use OliverKlee\Shop\Domain\Model\Article;

class ArticleTest extends \Nimut\TestingFramework\TestCase\UnitTestCase
{
    /**
     * @var Article;
     */
    protected $subject = null;

    protected function setUp()
    {
        $this->subject = new Article;
        $this->subject->initializeObject();
    }

    /**
     * @test
     */
    public function getNameInitiallyReturnsEmptyString()
    {
        self::assertSame('', $this->subject->getName());
    }

    /**
     * @test
     */
    public function setNameSetsName()
    {
        $name = 'foo bar';

        $this->subject->setName($name);

        self::assertSame($name, $this->subject->getName());
    }

    // ...
}
\end{phpcode}


\pagebreak
\section{Tests ausführen}

\subsection{Nicht-TYPO3-Projekte}

\subsubsection{Auf der Kommandozeile}

\begin{bashcode}
vendor/bin/phpunit Tests/
\end{bashcode}

\subsubsection{In PhpStorm}
\begin{enumerate}
  \item Settings > Languages \& Frameworks > PHP > PHPUnit
  \item PHPUnit library > Use Composer autoloader
  \item PHPUnit library > Path to script: \texttt{vendor/autoload}
  \item OK
  \item auf den Order \texttt{Tests/} rechtsklicken (oder einen anderen Ordner oder eine Testdatei)
  \item Run 'Tests'
\end{enumerate}


\subsection{TYPO3-Core}

Im TYPO-Wiki gibt es Anleitungen, wie man die Unit-Tests\footnote{\url{https://wiki.typo3.org/Unit_Testing_TYPO3}} und die funktionale Tests\footnote{\url{https://wiki.typo3.org/Functional_testing}} des Core ausführt.


\subsection{TYPO3-Extensions}

\subsubsection{In PhpStorm}

\paragraph{Für eine existierende TYPO3-Installation im Composer-Modus:}

Bei diesem Ansatz werden alle installierten Extensions geladen, sodass ihr auch die Features der PHPUnit-Extension nutzen könnt.

\begin{enumerate}
  \item Settings > Languages \& Frameworks > PHP > PHPUnit
  \item PHPUnit library > Use Composer autoloader
  \item PHPUnit library > Path to script: \texttt{vendor/autoload} aus dem Document-Root der TYPO3-Installation
  \item OK
  \item Run > Edit Configurations
  \item Defaults > PHPUnit
  \item Use alternative configuration file: \texttt{…/vendor/nimut/testing-framework/res/Configuration/UnitTests.xml} bzw. \texttt{…/vendor/nimut/testing-framework/res/Configuration/FunctionalTests.xml}
  \item Command Line > Environment variables
  \item vier Variablen hinzufügen (nur notwendig für functionale Tests):
    \begin{itemize}
      \item \texttt{typo3DatabaseUsername}
      \item \texttt{typo3DatabasePassword}
      \item \texttt{typo3DatabaseHost} (normalerweise \texttt{localhost})
      \item \texttt{typo3DatabaseName}
    \end{itemize}
  \item auf den Order \texttt{Tests/} rechtsklicken (oder einen anderen Ordner oder eine Testdatei)
  \item Run 'Tests'
\end{enumerate}


\paragraph{Für eine existierende TYPO3-Installation im Klassik-Modus (Nicht-Composer-Modus):}

In diesem Fall werdet ihr keine Klassen aus anderen Extensions autoloaden können, d.\,h., ihr werdet auch keine Features der PHPUnit-Extension nutzen können (und auch nicht von anderen Extension-Abhängigkeiten).

\begin{enumerate}
  \item Wenn ihr den TYPO3-Source per git statt als TAR-Paket heruntergeladen habt, braucht ihr ein \texttt{composer install} im TYPO3-Source-Verzeichnis.
  \item Settings > Languages \& Frameworks > PHP > PHPUnit
  \item PHPUnit library > Use Composer autoloader
  \item PHPUnit library > Path to script: \texttt{vendor/autoload} innerhalb des TYPO3-Source
  \item OK
  \item Run > Edit Configurations
  \item Defaults > PHPUnit
  \item Use alternative configuration file: \texttt{…/vendor/nimut/testing-framework/res/Configuration/UnitTests.xml} bzw. \texttt{…/vendor/nimut/testing-framework/res/Configuration/FunctionalTests.xml}
  \item Command Line > Environment variables
  \item vier Variablen hinzufügen (nur notwendig für functionale Tests):
    \begin{itemize}
      \item \texttt{typo3DatabaseUsername}
      \item \texttt{typo3DatabasePassword}
      \item \texttt{typo3DatabaseHost} (normalerweise \texttt{localhost})
      \item \texttt{typo3DatabaseName}
    \end{itemize}
  \item OK
  \item auf den Order \texttt{Tests/} rechtsklicken (oder einen anderen Ordner oder eine Testdatei)
  \item Run 'Tests'
\end{enumerate}


\paragraph{Ohne eine existierende TYPO3-Installation}

Dieser Ansatz benutzt das TYPO3-Extension-Skelett\footnote{\url{https://github.com/helhum/ext_scaffold}} von Helmut Hummel und Nicole Cordes.

Fügt die folgenden Abschnitte zur \texttt{composer.json} eurer Extension hinzu:

\small
\begin{jsoncode}
"require": {
  "typo3/cms": "~7.6.0"
},
"require-dev": {
  "namelesscoder/typo3-repository-client": "^1.2",
  "nimut/testing-framework": "^2.0",
  "phpunit/phpunit": "^5.7",
  "mikey179/vfsstream": "^1.6"
},
"config": {
  "vendor-dir": ".Build/vendor",
  "bin-dir": ".Build/bin"
},
"scripts": {
  "post-autoload-dump": [
    "mkdir -p .Build/Web/typo3conf/ext/",
    "[ -L .Build/Web/typo3conf/ext/tea ] || ln -snvf ../../../../. .Build/Web/typo3conf/ext/tea"
  ]
},
"extra": {
  "typo3/cms": {
    "cms-package-dir": "{$vendor-dir}/typo3/cms",
    "web-dir": ".Build/Web"
  }
}
\end{jsoncode}
\normalsize

Ersetzt dabei \texttt{tea} in Zeile 18 durch den Schlüssel eurer Extension.

Wenn ihr außerdem noch andere, im TER verfügbarer Extensions in den Tests nutzen möchtet (z.\,B.~die PHPUnit-Extension), fügt noch diese Abschnitte zu eurer \texttt{composer.json} hinzu (bzw.~ mergt sie):

\begin{jsoncode}
"repositories": [
{
  "type": "composer",
  "url": "https://composer.typo3.org/"
}
],
"require-dev": {
  "typo3-ter/phpunit": "*"
},
\end{jsoncode}

Führt danach diese Schritte in PhpStorm aus:
\begin{enumerate}
  \item Settings > Languages \& Frameworks > PHP > PHPUnit
  \item PHPUnit library > Use Composer autoloader
  \item PHPUnit library > Path to script:
    \begin{itemize}
      \item auf den Button \emph{Show hidden files and directories} klicken
      \texttt{.Build/vendor/autoload.php} im Verzeichnis der Extension
    \end{itemize}
  \item OK
  \item Run > Edit Configurations
  \item Defaults > PHPUnit
  \item Command Line > Environment variables
  \item vier Variablen hinzufügen (nur notwendig für functionale Tests):
    \begin{itemize}
      \item \texttt{typo3DatabaseUsername}
      \item \texttt{typo3DatabasePassword}
      \item \texttt{typo3DatabaseHost} (normalerweise \texttt{localhost})
      \item \texttt{typo3DatabaseName}
    \end{itemize}
  \item auf den Order \texttt{Tests/} rechtsklicken (oder einen anderen Ordner oder eine Testdatei)
  \item Run 'Tests'
\end{enumerate}



\pagebreak
\section{Mocks}

\subsection{Warum mocken?}
\begin{itemize}
  \item um eine Methode \glqq auszuschalten\grqq\ (damit sie nicht in die DB schreibt, kein Cruise-Missile abschießt etc.) und \texttt{null} zurückzugeben
  \item um einer Methode einen bestimmten Rückgabewert zu geben oder sie eine Exception werfen zu lassen
  \item um zu testen, dass eine Methode auf eine bestimmte Art und Weise aufgerufen wird
\end{itemize}

\subsection{Tools für Mocks}

\paragraph{Prophecy:} Das empfohlene, einfach benutzbare, aktuelle Mocking-Framework. Allerdings kann es keine partiellen Mocks erzeugen.\footnote{Prophecy-Cheatsheet:\\ \url{https://github.com/oliverklee/tdd-reader/blob/master/AdditionalDocuments/prophecy-cheatsheet.pdf}}
\paragraph{PHPUnit-Mocks:} Die alte Art, Mocks zu erzeugen. Mocking ist damit etwas unhandlich, aber dafür kann es auch partielle Mocks erzeugen.\footnote{PHPUnit-Mocking-Cheatsheet:\\ \url{https://github.com/oliverklee/tdd-reader/blob/master/AdditionalDocuments/mocking-cheatsheet.pdf}}
\paragraph{Mockery:} Auch sehr elegant.\footnote{\url{https://github.com/mockery/mockery}}



\pagebreak
\section{Auf Exceptions testen}

Laut einem Blogpost\footnote{\url{https://thephp.cc/news/2016/02/questioning-phpunit-best-practices}} von Sebastian Bergmann (dem Autor von PHPUnit) ist dies die aktuelle empfohlene Praxis.

\subsection{Nur auf die Klasse testen}
\begin{phpcode}
/**
 * @test
 */
public function createBreadWithNegativeSizeThrowsException()
{
    $this->expectException(\UnexpectedValueException::class);

    $this->subject->createBread(-1);
}
\end{phpcode}

\subsection{Auf Klasse, Nachricht und Code testen}
\begin{phpcode}
/**
 * @test
 */
public function createBreadWithNegativeSizeThrowsException()
{
    $this->expectException(\UnexpectedValueException::class);
    $this->expectExceptionMessage('$size must be > 0.');
    $this->expectExceptionCode(1323700434);

    $this->subject->createBread(-1);
}
\end{phpcode}


\section{Protected- und Private-Methoden testen}

\subsection{Indirekt testen (empfohlen)}

Tests sollten nur Verhalten testen, das von außen sichtbar ist. Daher solltet ihr Protected-Methoden über Test für Public-Methoden testen, die Gebrauch von den Protected-Methoden machen.

\subsection{Testing-Unterklasse erstellen}

Wenn ihr Protected-Methoden habt, die explizit Teil der API für Unterklassen sind, könnt ihr in \texttt{Fixtures/} eine Testing-Unterklasse erstellen, die eine Public-Methode hat, die die entsprechende Protected-Methode aufruft.


\pagebreak
\section{Abstrakte Klassen testen}

\subsection{Den PHPUnit-Mock-Builder benutzen}

Dies erzeugt eine Instanz der abstrakten Klassen, wobei alle abstrakten Methoden gemockt werden.

\begin{phpcode}
namespace OliverKlee\Coffee\Tests\Unit\Domain\Model;

use OliverKlee\Coffee\Domain\Model\AbstractBeverage;

class AbstractBeverageTest
{
    /**
     * @var AbstractBeverage|\PHPUnit_Framework_MockObject_MockObject
     */
    protected $subject = null;

    protected function setUp()
    {
        $this->subject = $this->getMockForAbstractClass(
            AbstractBeverage::class
        );
    }
\end{phpcode}

\subsection{Eine konkrete Unterklasse erstellen}

Dies wird empfohlen, wenn die Subklasse zusätzliches oder spezifisches Verhalten haben soll.

Erzeugt in \texttt{Tests/Unit/Unit/Domain/Model/Fixtures/} eine Unterklasse der abstrakten Klasse:

\begin{phpcode}
namespace OliverKlee\Coffee\Tests\Unit\Domain\Model\Fixtures;

class TestingBeverage extends \OliverKlee\Coffee\Domain\Model\AbstractBeverage
{
    // ...
}
\end{phpcode}

Dann könnt ihr die konkrete Unterklasse in euren Unit-Tests usen und instanziieren.

\begin{phpcode}
use OliverKlee\Coffee\Tests\Unit\Domain\Model\Fixtures\TestingBeverage;

class AbstractBeverageTest
{
    /**
     * @var TestingBeverage
     *
    protected $subject = null;

    protected function setUp()
    {
        $this->subject = new TestingBeverage();
    }
\end{phpcode}


\pagebreak
\section{Das Test-Framework der PHPUnit-TYPO3-Extension benutzen}

\begin{phpcode}
class DataMapperTest extends \Tx_Phpunit_TestCase
{
    /**
     * @var \Tx_Phpunit_Framework
     */
    protected $testingFramework = null;

    protected $subject = null;

    protected function setUp()
    {
        $this->testingFramework = new \Tx_Phpunit_Framework('tx_oelib');

        $this->subject = new ...;
    }

    protected function tearDown()
    {
        $this->testingFramework->cleanUp();
    }

    /**
     * @test
     */
    public function findWithUidOfExistingRecordReturnsModelDataFromDatabase()
    {
        $title = 'foo';
        $uid = $this->testingFramework->createRecord(
            'tx_oelib_test', ['title' => $title]
        );

        self::assertSame($title, $this->subject->find($uid)->getTitle());
    }
\end{phpcode}

\pagebreak
\section{Gemockte Dateisystem mit vfsStream benutzen}

\subsection{Lauffähige Beispiele}

Die funktionalen Tests zur FileUtility-Klasse im Tea-Beispiel\footnote{\url{https://github.com/oliverklee/ext_tea}} zeigen, wie Tests mit vfsStream aussehen können.


\subsection{Einrichten}
\begin{phpcode}
use org\bovigo\vfs\vfsStream;
use org\bovigo\vfs\vfsStreamDirectory;

/**
 * @var \org\bovigo\vfs\vfsStreamFile
 */
protected $moreStuff;

protected function setUp()
{
    // This is the same as ::register and ::setRoot.
    $this->root = vfsStream::setup('home');
    $this->targetFilePath = vfsStream::url('home/target.txt');

    $this->subject = new ...
}
\end{phpcode}

\subsection{Die Dateien benutzen}
\begin{phpcode}
/**
 * @test
 */
public function concatenateWithOneEmptySourceFileCreatesEmptyTargetFile()
{
    // This is one way to create a file with contents, using PHP's file functions.
    $sourceFileName = vfsStream::url('home/source.txt');
    // Just calling vfsStream::url does not create the file yet.
    // We need to write into it to create it.
    file_put_contents($sourceFileName, '');

    $this->subject->concatenate($this->targetFilePath, [$sourceFileName]);

    self::assertSame('', file_get_contents($this->targetFilePath));
}

/**
 * @test
 */
public function concatenateWithOneFileCopiesContentsFromSourceFileToTargetFile()
{
    // This is vfsStream's way of creating a file with contents.
    $contents = 'Hello world!';
    $sourceFileName = vfsStream::url('home/source.txt');
    vfsStream::newFile('source.txt')->at($this->root)->setContent($contents);

    $this->subject->concatenate($this->targetFilePath, [$sourceFileName]);

    self::assertSame($contents, file_get_contents($this->targetFilePath));
}
\end{phpcode}



\pagebreak
\section{Extbase-Controller testen}

Bei sauber gebauten Controllern werdet ihr hauptsächlich testen, dass die Daten korrekt vom Request zu den Repositories gehen sowie von den Repositories zum View. Außerdem werdet ihr gelegentlich die Forwards und Redirects testen wollen.

\subsection{Repositories mocken und injecten}

\begin{phpcode}
protected function setUp()
{
    $this->subject = new TestimonialController();

    $this->viewProphecy = $this->prophesize(ViewInterface::class);
    $this->view = $this->viewProphecy->reveal();
    $this->inject($this->subject, 'view', $this->view);

    $this->testimonialRepositoryProphecy
        = $this->prophesize(TestimonialRepository::class);
    $this->testimonialRepository
        = $this->testimonialRepositoryProphecy->reveal();
    $this->inject(
        $this->subject,
        'testimonialRepository',
        $this->testimonialRepository
    );
}
\end{phpcode}

Die \texttt{inject}-Method ist eine Hilfsmethode der Testcase-Basisklasse.

Wenn ihr für bessere Performance im Controller explizite \texttt{inject…}-Methoden benutzt statt der \texttt{@inject}-Annotationen, könnt ihr diese benutzen:

\begin{phpcode}
$this->subject->injectTestimonialRepository($this->testimonialRepository);
\end{phpcode}


\subsection{Datenfluss vom Repository zum View testen}

\begin{phpcode}
/**
  * @test
  */
public function indexActionPassesAllTestimonialsAsTestimonialsToView()
{
    $allTestimonials = new ObjectStorage();
    $this->testimonialRepositoryProphecy->findAll()
        ->willReturn($allTestimonials);

    $this->viewProphecy->assign('testimonials', $allTestimonials)
        ->shouldBeCalled();

    $this->subject->indexAction();
}
\end{phpcode}



\pagebreak
\section{PHPUnit-Assertions}
This list is current for PHPUnit 6.3.x.

\begin{verbatim}
assertArrayHasKey()
assertClassHasAttribute()
assertArraySubset()
assertClassHasStaticAttribute()
assertContains()
assertContainsOnly()
assertContainsOnlyInstancesOf()
assertCount()
assertDirectoryExists()
assertDirectoryIsReadable()
assertDirectoryIsWritable()
assertEmpty()
assertEqualXMLStructure()
assertEquals()
assertFalse()
assertFileEquals()
assertFileExists()
assertFileIsReadable()
assertFileIsWritable()
assertGreaterThan()
assertGreaterThanOrEqual()
assertInfinite()
assertInstanceOf()
assertInternalType()
assertIsReadable()
assertIsWritable()
assertJsonFileEqualsJsonFile()
assertJsonStringEqualsJsonFile()
assertJsonStringEqualsJsonString()
assertLessThan()
assertLessThanOrEqual()
assertNan()
assertNull()
assertObjectHasAttribute()
assertRegExp()
assertStringMatchesFormat()
assertStringMatchesFormatFile()
assertSame()
assertStringEndsWith()
assertStringEqualsFile()
assertStringStartsWith()
assertThat()
assertTrue()
assertXmlFileEqualsXmlFile()
assertXmlStringEqualsXmlFile()
assertXmlStringEqualsXmlString()
\end{verbatim}

\bibliography{tdd-reader}

\end{document}
